\section{Literature Review}
This PhD is built upon the previous research that I conducted during my Honours, and there is significant overlap with the two topics.
Therefore, the first item that is worthy of mention would be `Harmonic Based Extended Techniques and their Compositional Applications', which includes a ground-level review of the most seminal literature in the field.
However, due to the limited scope of the exegesis, there were significant ommissions, so for the sake of completeness its contents will be reviewed under the lens of `composing with extended techniques'.

To understand extended techniques, we must first establish a basis of what is considered a `regular' technique, or rather, what the qualifying factors are for a technique to be considered extended.
To do this, we will look at what techniques are commonplace in the literature, and which are less so.
Techniques that require descriptions in the frontmatter or otherwise `extend' the instrument beyond the normal canon would reasonably be understood to be considered extended techniques.



Robert Dick's seminal `The Other Flute' was a landmark work in extended technique documentation, painstakingly notating the outputs, fingerings, and qualities of flute multiphonics and other extended techniques.\autocite{dickOtherFlute1989}
The structure of its content is logical, though is hampered by the need to refer to a key.

\fxnote{This probably needs retooling to fit better in the literature review.}There are three aspects that are relevant to the literature review; literature surrounding the techniques themselves, literature with how the techniques are presented, and scores. 
Scores are necessary to understand how composers implement extended techniques in practice.
By reviewing their frontmatter and the symbol notation, we gain an understanding of how composers present information to performers, and can build a system of notation that maps consistently with the rest of the established canon.
Consequently, seminal scores will be included in the literature review so that we may understand how they work. 
These scores' notation systems for extended techniques will be broken down into their elements, and through comparative analysis of how different composers implement the same techniques, we will be able to find how textual, symbolic, and graphic notation systems can be used to achieve the desired results.

Textual notation, i.e. instructions printed in the score, are the most straight forward, but limited to what can be summarised in few words.
Symbolic notation assigns the technique to a symbol, typically with the instructions placed in the frontmatter.
Graphical notation systems are often used when the binary of symbolic notation is restrictive, and requires a greater fidelity than textual notation can provide. 
An example of this can be seen in Kaija Saariaho's notation of overpressure, wherein she uses a black bar to represent the amount of overpressure required, temporally relational to the position in the score.
Discussion of the framework used in Dimpker's thesis.\autocite{dimpkerExtendedNotationDepiction2012}

The most significant of developments in the area is the recent special January 2020 Tempo journal issue, dedicated entirely to string multiphonics.
It was collated by Dr. Ellen Fallowfield, who notably contributed the landmark thesis CelloMap (and eponymous website) in 2009.\autocite{fallowfieldCelloMapHandbook2009}