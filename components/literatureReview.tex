\section{Literature Review}
This PhD continues upon the previous research that I conducted during my Honours, and there is significant overlap with the two topics.
Therefore, the first item that is worthy of mention would be `Harmonic Based Extended Techniques and their Compositional Applications', which includes a ground-level review of the seminal literature in the field.\autocite{gray}
However, due to the limited scope of the exegesis, there were significant omissions, so for the sake of completeness its contents will be reviewed under the lens of `composing with extended techniques'.
Additionally, there are a number of sources that were either missed or cut from the initial literature review for sake of brevity.
There appears to be a great deal of activity in similar spheres of research in Basel, Switzerland, and there are several manuscripts and texts that appear only in German.

Unlike standard pedagogical models, wherein the student learns from the teacher, the methods proposed in this are a more collaborative, and less rigid practice.
Composers typically will have a performer in mind when they compose a work, and collaborate with them, working to find what works for the artists and instrument.
However, in subsequent performances, that direct connection to the composer is often unavailable, and the performer is left to interpret the paratext, without direct instruction from the composer.
Paratextual instruction can vary in degrees of specificity, and will often be printed in the frontmatter of the work.
In some cases, though, the performer must rely on previous works to build a contextual language in which to interpret esoteric markings, or use previous recordings of interviews and performances to ascertain the composer's intent.
Understanding the composer's intent is a crucial part in the preparation process, and the ability to accurately reproduce the composers' intent may negatively impact a work's longevity in the literature.
Many academics have recognised the deficit in training in extended techniques, both in their base form and their composer-specific implementation (which may vary from composer to composer, or piece to piece). Violist Sarah Wei-Yan Kwok discusses this in her thesis `Breaking the sound barriers : extended techniques and new timbres for the developing violist', where she commissions six etudes for viola exploring extended techniques, along with an investigation into pedagogy.\autocite{kwokBreakingSoundBarriers2018}

\begin{quotation}
    Many so-called ‘new’ instrumental devices have developed from well established techniques; they are extensions of, or refinements of, procedures long considered part of a composer’s repertorium of expressive devices. The newness, then, is not one of kind but of degree, a further and more extensive development of basic effects found in scores from the late nineteenth century to the present day. 
    % TODO:Cite Read, Contemporary Instrumental Techniques, 3.
\end{quotation}

Unpacking the language that we use, extended techniques are just that; an extension of the traditional techniques that are already established in the canon. 

\subsection{Understanding Technique}

\subsubsection{Standard Technique}



\paragraph{Pedagogy}

\subsubsection{Extended Techniques}

\subsection{}
To understand extended techniques, we must first establish a basis of what is considered a `regular' technique, or rather, what the qualifying factors are for a technique to be considered extended.
To do this, we will look at what techniques are commonplace in the literature, and which are less so.
Techniques that require descriptions in the frontmatter or otherwise `extend' the instrument beyond the normal canon would reasonably be understood to be considered extended techniques.



Robert Dick's seminal `The Other Flute' was a landmark work in extended technique documentation, painstakingly notating the outputs, fingerings, and qualities of flute multiphonics and other extended techniques.\autocite{dickOtherFlute1989}


\fxnote{This probably needs retooling to fit better in the literature review.}There are three aspects that are relevant to the literature review; literature surrounding the techniques themselves, literature with how the techniques are presented, and scores. 
Scores are necessary to understand how composers implement extended techniques in practice.
By reviewing their frontmatter and the symbol notation, we gain an understanding of how composers present information to performers, and can build a system of notation that maps consistently with the rest of the established canon.
Consequently, seminal scores will be included in the literature review so that we may understand how they work. 
These scores' notation systems for extended techniques will be broken down into their elements, and through comparative analysis of how different composers implement the same techniques, we will be able to find how textual, symbolic, and graphic notation systems can be used to achieve the desired results.

Dimpker defines 

Textual notation, i.e.\ instructions printed in the score, are the most straight forward, but limited to what can be summarised in few words.
Symbolic notation assigns the technique to a symbol, typically with the instructions placed in the frontmatter.
Graphical notation systems are often used when the binary of symbolic notation is restrictive, and requires a greater fidelity than textual notation can provide. 
An example of this can be seen in Kaija Saariaho's notation of overpressure, wherein she uses a black bar to represent the amount of overpressure required, temporally relational to the position in the score.
Discussion of the framework used in Dimpker's thesis.\autocite[23]{dimpkerExtendedNotationDepiction2012}

The most significant of developments in the area is the recent special January 2020 Tempo journal issue, dedicated entirely to string multiphonics.
It was collated by Dr.\ Ellen Fallowfield, who notably contributed the landmark thesis CelloMap (and eponymous website) in 2009.\autocite{fallowfieldCelloMapHandbook2009}