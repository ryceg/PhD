\section{Literature Review}
This PhD continues upon the previous research that I conducted during my Honours, and there is significant overlap with the two topics.
Therefore, the first item that is worthy of mention would be `Harmonic Based Extended Techniques and their Compositional Applications', which includes a ground-level review of the seminal literature in the field.\autocite{grayHarmonicBasedExtended2019}
However, due to the limited scope of the exegesis, there were significant omissions, so for the sake of completeness its contents will be reviewed under the lens of `composing with extended techniques'.
Additionally, there are a number of sources that were either missed or cut from the initial literature review for sake of brevity.
% There appears to be a great deal of activity in similar spheres of research in Basel, Switzerland, and there are several manuscripts and texts that appear only in German.

Unlike standard pedagogical models, wherein the student learns from the teacher, the methods used in this exegesis are a more collaborative, and less rigid practice.
Composers typically will have a performer in mind when they compose a work, and collaborate with them, working to find what works for the artists and instrument.
However, in subsequent performances, that direct connection to the composer is often unavailable, and the performer is left to interpret the paratext, without direct instruction from the composer.
Paratextual instruction can vary in degrees of specificity, and will often be printed in the frontmatter of the work.
Where there is a well established understanding of the technique (such as the Bartok pizzicato technique, which has been accepted into the canon of `standard' techniques, and is no longer considered extended), this is not an issue, but can pose issues in the case of techniques where there is a great deal of variance possible, such as multiphonics and subharmonics.
In some cases, the performer must rely on previous works to build a contextual language in which to interpret esoteric markings, or use previous recordings of interviews and performances to ascertain the composer's intent.
Understanding the composer's intent is a crucial part in the preparation process, and the ability to accurately reproduce the composer's intent may negatively impact a work's longevity in the literature.
Many academics have recognised the deficit in training in extended techniques, both in their base form and their composer-specific implementation (which may vary from composer to composer, or piece to piece). 
Violist Sarah Wei-Yan Kwok discusses this in her thesis `Breaking the sound barriers: extended techniques and new timbres for the developing violist', where she commissions six etudes for viola exploring extended techniques, along with an investigation into pedagogy.\autocite{kwokBreakingSoundBarriers2018}

The composer-performer collaboration dynamic has been explored at length; the most famous example being the virtuoso pianist Paul Wittgenstein, whose career came to a halt during World War 1, when he lost his right arm.
Afterwards, he began commissioning leading composers to write for him, resulting in Richard Strauss', `Parergon zur `, `Sinfonia Domestica for Piano and Orchestra', Maurice Ravel's `Piano Concerto for the Left Hand in D major', and further works written by Hindemith, Korngold, Britten, and Prokofiev, to name a few.\autocite[107]{predotaPaulWittgensteinVoice2014}
Collaboration with the intended performer is crucial to a composer's ideas being realised accurately, as articulated by Jack Barnes, who stated:
\begin{quotation}
    My collaboration with Mathieson-Sandars allowed for subtle but important improvements to \emph{Lines, Contexts and Freedoms}. 
    Certain aspects such as the composer hearing his piece for the first time, experimenting with different pianistic timbres and creating more comfortable hand distributions were among the most useful outcomes of our collaboration. 
    As the performer, it was useful for my interpretation to learn [the composer's] intent behind the gestures; that the effect of dynamic shaping was not desirable for all of them.\autocite[20]{jackbarnesExaminationComposerPerformerCollaborations2017}
\end{quotation}
Discussion being needed with the composer to learn the composer's intent suggests that there was information being provided to Barnes that was not present in the paratext.
Future performers of the work that was commissioned by Barnes may not have the same level of access to the composer, hence the need for a thorough and precise documentation of the intended sound being available.

Sheet music is an abstracted conception of a musical work; the printed paper is merely a set of instructions that are interpreted by the performer, in order to express a musical idea that only the composer truly knows.
Without a perfect abstraction of what the composer imagines, a `perfect' reproduction is impossible.
Indeed, even if there \emph{was} a perfect reproduction, the variance in instrumental timbre and other facets of the process render it a fool's errand.
The Platonic ideal is unattainable, and perhaps why little effort has been made to make further progress towards the liminal.\autocite{citation needed, obviously}

Attempts at combining notation with composer intent have been made, with Eftiha Victoria Arkoudis' `Contemporary Music Notation for the Flute: A Unified Guide to Notational Symbols for Composers and Performers' building on the work of Robert Dick.\autocite{arkoudisContemporaryMusicNotation2019}

Instructive manuals are typically in one of two categories; performer-oriented, in which the techniques presented are articulated in a performer-focused context, and composer-oriented; orchestration manuals which dictate the end results.
For brevity, these usually do not intersect, or at least are not written comprehensively with both the composer and performer in mind. 
The lack of literature that encompasses the entirety of the production of sound process, from the action to the resultant, is slowly being rectified, particularly with online resources such as Heather Roche's work in clarinet multiphonics, and Fallowfield's with CelloMap gaining popularity.\autocite{rocheHeatherRoche, fallowfieldCelloMap}
\subsubsection{Notation Manuals}
Notation has long been the domain of Kurt Stone and Gardner Read being the authoritative voices, writing extensively on the subject beginning in the 1970s.\autocite{stoneMusicNotationTwentieth1980,readCompendiumModernInstrumental1993,readContemporaryInstrumentalTechniques1976,readMusicNotationManual1979a}
Additional texts by David Cope and Erhard Karkoschka supported these with their own contributions.\autocite{copeNewMusicNotation1976,karkoschkaNotationNewMusic1972}
Editor of Faber Music, Elaine Gould's seminal `Behind Bars' gave an authorative voice to the methods of notating along with the building blocks for techniques which she felt had not been developed to the point of consensus.\autocite[iii]{gouldBars2011}

\subsection{Physics}
In order to be able to accurately prescribe instructions on how to produce these techniques, there must be an understanding of the underlying physics.
Helmholtz provides our basic undestanding of the stick---slip motion which makes a bow produce sound on a string.\autocite{helmholtzSensationsTonePhysiological1954}
Knut Guettler and Håkon Thelin have provided extensive research into how the string can produce multiphonics.\autocite{thelinAnalysisBowedStringMultiphonics,thelinMultiphonicsDoubleBass2011,guettlerBowedstringMultiphonicsAnalyzed2012,guettlerGuideAdvancedModern1992,guettlerWaveAnalysisString1994}
\fxnote{This needs to be expanded}{R. T. Schumacher, Kenneth Marskall, and Anders Askenfelt contribute additional aspects to the literature.}\autocite{schumacherTransientBehaviourModels1995,askenfeltMeasurementBowingParameters1989,marshallModalAnalysisViolin1985}
For subharmonics, the violinist that discovered the technique, Mari Kimura, has contributed practical examples on how to produce them.\autocite{kimuraHowProduceSubharmonics1999}
\fxnote{Empirical data on what? Expand!}{However, for research into the technique's production, Robert Mores, Guettler, Erwin Schoonderwalt, Carleen Hutchins, and John Cantrell provide empirical data.}\autocite{moresFurtherEmpiricalData2019,guettlerBowedStringDevelopment2002,schoonderwaldtViolinistSoundPalette2009,hutchinsSubharmonicsPlateTap1960,cantrellSubharmonicGenerationChaos2015}

\subsection{Extended Techniques}
To understand extended techniques, we must first establish a basis of what is considered a `regular' technique, or rather, what the qualifying factors are for a technique to be considered extended.
To do this, we will look at what techniques are commonplace in the literature, and which are less so.
Techniques that require descriptions in the frontmatter or otherwise `extend' the instrument beyond the normal canon would reasonably be understood to be considered extended techniques.
Read discusses this, stating:
\begin{quotation}
    Many so-called ‘new’ instrumental devices have developed from well established techniques; they are extensions of, or refinements of, procedures long considered part of a composer’s repertorium of expressive devices. 
    The newness, then, is not one of kind but of degree, a further and more extensive development of basic effects found in scores from the late nineteenth century to the present day.\autocite[3]{readContemporaryInstrumentalTechniques1976}
\end{quotation}
Unpacking the language that we use, extended techniques are just that; an extension of the traditional techniques that are already established in the canon. 

\fxnote{This probably needs retooling to fit better in the literature review.}There are three aspects that are relevant to the literature review; literature surrounding the techniques themselves, literature with how the techniques are presented, and scores. 
Scores are necessary to understand how composers implement extended techniques in practice.
By reviewing their frontmatter and the symbol notation, we gain an understanding of how composers present information to performers, and can build a system of notation that maps consistently with the rest of the established canon.
Consequently, seminal scores will be included in the literature review so that we may understand how they work. 
These scores' notation systems for extended techniques will be broken down into their elements, and through comparative analysis of how different composers implement the same techniques, we will be able to find how textual, symbolic, and graphic notation systems can be used to achieve the desired results.

Dimpker postulates the following criteria for extended technique notation; \begin{quote}
    `(1.) As exact as possible and (2.) As simple as possible while the system may (3.) Not be contradictory to
traditional notation, but should instead extend and be closely related to it.'\autocite[3]{dimpkerExtendedNotationDepiction2012}
\end{quote}
This meshes well with the philosophy of Elaine Gould's `Behind Bars', which advocates for a consistent style language, despite the two texts occasionally coming at odds with one another in the exact treatment of some cases.\autocite[120--121, 61]{dimpkerExtendedNotationDepiction2012,gouldBars2011}
Dimpker posits that four methods of notation are acceptable in order to realise exact and precise notation; `action notation, symbolic notation, diagrammatic notation, and schematic notation'.\autocite[33]{dimpkerExtendedNotationDepiction2012}

Textual notation, i.e.\ instructions printed in the score, are the most straight forward, but limited to what can be summarised in few words.
Symbolic notation assigns the technique to a symbol, typically with the instructions placed in the frontmatter.
Graphical notation systems are often used when the binary of symbolic notation is restrictive, and requires a greater fidelity than textual notation can provide. 
An example of this can be seen in Kaija Saariaho's notation of overpressure, wherein she uses a black bar to represent the amount of overpressure required, temporally relational to the position in the score.\autocite{TODO:saariaho citation}

\subsubsection{Double Touch Harmonics}
The technique of `double touch harmonics' as described by John Addison appears to be almost totally novel. 
Chase T Jordan references the technique briefly as `dual node harmonics', but provides only a basic overview of their treatment.\autocite[36]{jordanOldInstrumentsNew2020}
The only other reference that seems to be relevant is a 1980 thesis, which has not been made available online.\autocite{woodrichMultinodalPerformanceTechnique1980}
Addison has completed several fingering charts and made preliminary instructive documentation on the technique, which has been provided to the researched.
Due to the lack of literature, collaboration with performers will be essential in order to build the technique to a stable, usable state.

\subsubsection{Subharmonics}
No significant output in the research of the mechanics of subharmonics has occurred since the publication of my Honours exegesis, though there have been several works composed which make use of subharmonics.\

\subsubsection{Multiphonics}
Multiphonics have been well-documented in reed instruments, with Robert Dick's seminal `The Other Flute', setting the gold standard in extended technique documentation, painstakingly notating the outputs, fingerings, and qualities of flute multiphonics and other extended techniques.\autocite{dickOtherFlute1989}
There are several Barenreiter technique manuals for other aerophones that cover multiphonics, including bassoon, saxophone, violin, and oboe, with the publisher appearing to try and compile a manual for every instrument.\autocite{weissTechniquesSaxophonePlaying2010,galloisTechniquesBassoonPlaying2009,ardittyTechniquesViolinPlaying2013, TODO:OboeBook}
\fxnote{Double check to see if this is true}The book on violin by violinist Irvine Arditty notably lacks any information on multiphonics, perhaps due to the difficulty of production on the small instrument.\autocite{ardittyTechniquesViolinPlaying2013}

\paragraph{String instruments}
The development of literature dealing with multiphonics on stringed instruments is more recent, with a special January 2020 Tempo journal issue, dedicated entirely to string multiphonics.
It was collated by Dr.\ Ellen Fallowfield, who notably contributed the landmark thesis CelloMap (and eponymous website) in 2009.\autocite{fallowfieldCelloMapHandbook2009, fallowfieldCelloMap}
Her article, `Cello Multiphonics: Technical And Musical Parameters' expands upon the work that she began in her thesis, `CelloMap'.\autocite{fallowfieldCelloMultiphonicsTechnical2020}

\paragraph{Plucked chordophones}
Rita Torres has explored multiphonics on the guitar extensively, with several papers dedicated to their research, including her thesis, `A New Chemistry Of Sound: The Technique Of Multiphonics As A Compositional Element For Guitar And Amplified Guitar'.\autocite{torresMultiphonicsCompositionalElement2012}
Paulo Ferreira-Lopez and Torres provide a robust exploration of previous systems of notation of multiphonics, suggesting notational methods similar to those that Fallowfield suggests.\autocite{ferreira-lopesGuitarMultiphonicsNotations}
Catalogues of multiphonics in the guitar literature show that there has been little interest in the technique, with the first score dating back to 1832, but only around twenty or so pieces composed since the techniques inception.\autocite[80--82]{torresSoundWorldGuitar2018}
It has been suggested that the widespread adoption of the technique has been marred by the mode of playing naturally having a decay, making multiphonics more difficult to hear for an audience, as well as the guitar having a poor dynamic range.\autocite[21--22]{Torres2014TowardsOT}
This has been all but confirmed with the analysis of multiphonic signal decay at distances equivalent to that of an audience revealing that without amplification, the multiphonics will be inaudible.\autocite[279]{torresGuitarMultiphonicsInfluence2014}
Thomas Ciszak and Seth Josel expand upon the research carried out by Torres, collating a catalogue of performable multiphonics from strings 3 --- 1, and examining multiphonics on the electric guitar.\autocite{ciszakNeonLightMultiphonic2020}

\paragraph{Piano}
The piano, an unlikely candidate for multiphonics, has had some success in making inroads in the technique through the use of preparing the piano.
With the development of more robust literature, the prospect of developing a new sound-palette for the piano is alluring.
Interest begins in 2016 with Juhani Vessikala's MA thesis on the topic.\autocite{vesikkalaMultiphonicsGrandPiano2016}
Sanae Yoshida acknowledges Vessikala's work in an article in the multiphonics issue of Tempo, where Yoshida expands upon the practical ways in which a composer can use microtonality (and multiphonics) on a piano.\autocite{yoshidaMicrotonalPianoTunedIn2020}
Following is Caspar Walter's algorithm to calculate the frequency components of pure multiphonics.\autocite{casparjohanneswalterVariantsContinuedFraction2019}

\subsubsection{Conclusion}
It becomes clear that there is significant research that is being carried out in the realm of novel extended techniques, but there is still a deficit in resources for composers in implementing the techniques.
