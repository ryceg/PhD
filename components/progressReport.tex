\section{Progress Overview}
In July 2020, I began my PhD project titled `Writing Wrong Right: Composing with Extended Techniques', with Dr.\ Maria Grenfell and Dr.\ Sean Priest supervising.
My initial focus was on the collation of practical advice for composers seeking to write music incorporating extended techniques, but since then, has shifted topic, which I feel is both better suited to my skillset, and is more novel research.
The premise of the research now is that Western staff notation has been developed piece by piece, over hundreds of years, and lacks a formal schema that composers can use to construct new forms of notation and technique; 
This is especially urgent with the increasing uptake of new formats of engaging with scores reach critical mass (for example, the animated score was inaccessible and niche until iPads achieved mainstream adoption).
The construction of a `cookbook' that details the methods and elements of communicating musical ideas will assist adoption of new formats of presenting musical scores, future-proofing the format by genericizing actions, allowing composers to construct new notation in a logically consistent manner.
The works that I compose as the folio portion of this thesis will be practical examples of the application of this `cookbook', and demonstrate how composers can work within the Western staff notation system, or adapt it to fit their particular needs.
As an example, the work \emph{how to decay gracefully} (see 
% \autoref{app:howtodecaygracefullyscore}
the appendix for the score
) is the result of investigating how sections of music that are dictated by the rhythm of speech can be incorporated into Western staff notation.

To date, I have had one public performance of a work, composed two new works exploring the techniques that I have developed, and have been accepted to a composition workshop program in Sydney by Halcyon, a contemporary music ensemble.
Unfortunately, due to the Sydney lockdown, this has been postponed until restrictions ease, but will culminate in traveling to Sydney and recording the work that I compose.

I had initially planned to conduct interviews (and thus prepared an Ethics application to do so), but this was put to the side as the direction of the research pivoted.
However, upon re-review, I believe that there could be much to be gained from interviewing musicians to supplement the document-based analysis that forms the bulk of the research at the moment. 
As the original ethics application was low-risk and broad in its definitions, it is not anticipated that there will be many alterations needed to make the application fit the new scope of research.

I am pleased with the progress that I have made, and am much more confident in the direction that the enquiry has taken; the initial research direction was retreading old ground of orchestration, rather than examining novel concepts.
I think that the outputs of this research will be of benefit both to my own personal artistic development, and the finished thesis will be an asset to the larger compositional community.
The interdisciplinary nature of this research, where it intersects with semiotics, ethnosemantics, and engages with new technologies makes it of interest to multiple stakeholders.
The componential analysis of signs makes this research especially relevant to the education sector.
It should be noted that in their 2021 curriculum review, the Australian Curriculum, Assessment and Reporting Authority has recommended a revision of the terminology used, replacing `notation' with `documentation', stating that \begin{quotation}
    The term ‘documentation’ is used rather than ‘notation’ to describe a record of heard or imagined music. 
    This change strengthens the curriculum’s accessibility to all students and its relevance across existing and future approaches to composition and performance. 
    Documentation can include notation forms such as Western staff, notations from classical music traditions, graphic, symbol-based or visual forms, tablature, audio recordings, video, animated or interactive notation.\autocite[6]{WhatHasChanged2021}
\end{quotation}
This signifies a clear shift towards a growing acceptance of different forms of representing music, a topic pertinent to this research, which seeks to lower the barrier of entry to extended techniques.