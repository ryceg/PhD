\section{Development of a Relevant Literature Database}

% consistent and accurate content and formatting within the citation fields
% • how new references are added (e.g. direct export from specific databases, typed
% manually, imports from alerts)
% • how candidates keep track of files related to the references (e.g. linking to
% PDFs)
% • how references are described or grouped to enable easy search and retrieval
% • the selection of bibliographic styles appropriate to the discipline or audience,
% used in a piece of writing


To develop a relevant literature database for my PhD `Writing Wrong Write: Investigating Composing with Extended Techniques', care must be taken to ensure that the data that I consolidate is preserved for future reference.
I plan to use the Turabian referencing system, as it is the standard used for the majority of music-related academic writing.
I use Zotero, a free and open source reference manager, which integrates into both Word, and \LaTeX, which is what I will be using to compile my exegesis.
Zotero has a multitude of benefits, including the ability to categorise references, tag them, store static PDFs, and automatically fill out data fields when provided with a URL, ISBN, DOI, PMID, or arXiv ID.
In addition to this, it can be hosted online, ensuring easy collaboration between myself and my supervisors. 
It additionally can store attachments and linked URLs, ensuring that I can keep track of files and material related to the reference.
A bibliography can be generated in the BibLaTeX file format, which integrates into LaTeX for easy compilation. 

When a new article or piece of literature is found, it is added to Zotero by its DOI, URL, or ISBN.
Zotero then searches for it, and if it finds the relevant document, autofills as much of the information as possible.
After double-checking to ensure that all of the information is correct, and none was missed, the item can be categorised and incorporated into the database.
Tags are applied to the item, and it is then placed in all relevant collections (of which there can be more than one).
Tagging and categorising into collections serve similar, but different purposes; collections are hierarchical, whereas tags are not, and can be filtered with operators.
Collections serve a more general purpose, but for more granular control, tagging is preferred. 

For retrieval of references inside the Zotero interface for finding relevant literature which is already part of the database, a user can navigate through the collections and subcollections, or apply tags to filter by.
Multiple tags can be selected in the same search, allowing for filtering for general topics, which will then reveal more specific ones, or vice versa. 
Once an item is found, the data stored is readily accessible, including the original place that the item was found, and the time of access. 
Any attachments or URLs linked to the item can then be accessed.

For compilation into an academic document using LaTeX, the user can cite the item, drawing upon the auto-generated works-cited.bib file using \begin{verbatim}
    \cite{}
\end{verbatim}.