\section{Brief Background}
In 2019, I completed my Honours exegesis at the University of Tasmania, titled `Harmonic Based Extended Techniques and their Compositional Applications', a study of three extended techniques applicable to stringed instruments; half-harmonics, subharmonics, and multiphonics.
I am interested in pursuing this line of research further, as I believe that there is still more to be learnt about the techniques that I have already covered, and that many other extended techniques lack the literature that composers can use to make informed decisions about their use. 
The purpose of this study will therefore be to explore extended techniques further, filling literature gaps and incorporating the techniques into my own artistic language.

% Cellist and new music specialist John Addison has expressed an interest in developing his technique of `double touch' harmonics with me, which he describes as `[\ldots] where one engages two of the harmonic nodes on the same string in the same series simultaneously, which allows upper harmonics in the series that have never been stable nor dependable to become reliable and certain.'
% This discovery expands exponentially not only the facility of producing the said harmonics but when applied to double-stops (the production of tones on two strings simultaneously) it increases the available options from around 40 to 243.


% His technical and theoretical knowledge will be useful in testing and conducting practical research on the techniques, with eminent composer Sofia Gubaidulina stating 
%     `I am convinced that we are dealing with a brilliant artistic personality here. 
%     It is to be expected that John Addison’s activities as an interpreter will have a vital influence on the next generation of musicians.'\footnote{Personal correspondance between John Addison and Sofia Gubaidulina.}
As a spectralist composer, I am primarily concerned with extended techniques that make use of microtonalities, exploit the harmonic series, and spatial acoustics.
% The recent shift of the Conservatorium of Music to the Hedberg provides an exciting opportunity to conduct research using the variable acoustic panels, and the ways that they can be used in site-specific works. 
The scope of my research would therefore be surrounding the treatment of these extended techniques; a holistic review of the techniques from the view of a composer and performers will shed light on the way that the techniques can best be produced.

As a composer, exploring subharmonics, multiphonics, double touch harmonics, and other extended techniques is particularly exciting, as they are fertile ground for new and unique sounds that can be used to develop my musical identity.
% I plan for the resultant exegesis to be a practical document that composers and artists can use as a reference manual for the production and implementation of the techniques in their own practices.
My research into underexplored techniques will broaden the performative and compositional palette available to artists. 
Through the documentation of my process in researching this technique, it will be catalogued and brought into the literature, facilitating further development.


Through a comprehensive review of how composers construct their frontmatter, guidelines to how new and experimental techniques can be communicated to performers will be developed.
This will lower the friction of learning new works, and promote the uptake of contemporary works.
This will be further aided by the development of a \LaTeX{} style which can quickly scaffold the relevant extended techniques for consistent and universal verbiage.

The resultant exegesis, `Writing Wrong Right: Composing With Extended Techniques', will consequently be a practical document, suitable as a reference for artists interested in implementing extended techniques into their practice.

