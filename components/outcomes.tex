\subsection{Outcomes Of This Project, and Why It Is Relevant}
% This project will provide me with a better understanding of the mechanics and musical capabilities of the extended techniques. 

% My research will address this, filling the research gaps where identified.
% The outcome of my research into the technique of `double touch harmonics', as developed by John Addison, is particularly relevant as it presents an exciting possibility for a new method of producing familiar harmonics.
% This will increase the number of available fingering positions of harmonics within existing compositions, as well as providing more colour options for performers to choose from.
There is a dearth of literature regarding how composers interact with the score, and communicate their ideas via the score to the performer.
Notation and implementations of techniques are all on an ad-hoc basis, with composers needing to reference specific books for aspects of notation, implementation, and communication of their ideas.
The philosophy of how intent is communicated is inaccessible to composers seeking to strike new ground, and this research will provide scaffolding upon which composers can fabricate new ideas.
As these techniques currently have a deficit of literature, both instructive and artistic; there are few resources for people to learn from, and even fewer practical examples of how to implement the techniques in a musical context.
My research will produce a practical document that outlines the methods of communicating intent to a performer, and my folio will comprise of examples so that others may observe how I develop new methods, using existing terminology, iconography, and semiotics where possible to lower the friction of uptake.