\subsection{Outcomes Of This Project, and Why It Is Relevant}
This project will provide me with a better understanding of the mechanics and musical capabilities of the extended techniques. As these techniques currently have a deficit of literature, both instructive and artistic; there are few resources for people to learn from, and even fewer practical examples of how to implement the techniques in a musical context.
My research will address this, filling the research gaps where identified.
The outcome of my research into the technique of `double touch harmonics', as developed by John Addison, is particularly relevant as it presents an exciting possibility for a new method of producing familiar harmonics.
This will increase the number of available fingering positions of harmonics within existing compositions, as well as providing more colour options for performers to choose from.
Through the documentation of my process in researching this technique, it will be catalogued and brought into the literature, facilitating further development.
This is just one example of how my research into underexplored techniques will broaden the performative and compositional palette available to artists. 
The resultant thesis, `Writing Wrong Right: Composing With Extended Techniques', will consequently be a practical document, suitable as a reference for artists interested in implementing extended techniques into their practice.



