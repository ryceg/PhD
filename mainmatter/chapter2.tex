\section{Methodology}

Cataloguing the entirety of the Western classical music canon of music notation is a monumental task, but can be broken into smaller, more manageable tasks by working from first principles.
If we limit our scope to interfaces that require interaction from humans to create the work, we immediately cut out swathes of computer-centric design patterns.
We can then further 

Through interviews with players at varying stages of proficiency and familiarity with the techniques, I will be able to uncover the barriers to producing these techniques. 
Document analysis of existing resources and compositions will help direct and support the line of enquiry. 
Autoethnography of my creative process will document the research process and clarify my intent.  

The aim of this research project is not to make the techniques popular enough to make clarification of technique unnecessary, or for it to enter the canon of techniques so that it is no longer considered to be `extended' (as the Bartok pizzicato has).
Rather, this is intended to act as a resource for composers and artists to be drawn upon as a reference for when they wish to use the technique.
A considered and informed judgement call over a technique can only be made when the technique is understood well.
The composer will communicate the information necessary to realise the technique to the player, typically through the frontmatter. 
In order to better understand what information composers deem useful to communicate to players, a review of scores with similar techniques will take place.
By breaking the score's frontmatter content up into its actions, we can understand how composers communicate their desired techniques to players.