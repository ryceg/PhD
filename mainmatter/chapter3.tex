% chapter3.tex (Chapter 3 of the thesis)

\chapter{Compositions and Implementation of Techniques}\label{ch:chapter3}
% In this chapter, I will present my folio and subsequent findings through the compositional process.
% This will be more broad, and I will make amendments to what I posited in Chapter 2. 
% \subsection{Background}
% Provide a better understanding on the ways that these techniques can be incorporated idiomatically into a composer's practice.

% \subsection{Research statement/problem}
% There is a dearth of resources for composers interested in these techniques, as evidenced by this research being novel.

% \subsection{Aim and scope of thesis}
% Ways to incorporate and present these techniques in an idiomatic way that is intuitive.

% \subsection{Significance of work}
% The properties of these techniques and the ways that they can be used idiomatically.

My folio of works comprises of four pieces: \nameref{sec:violaPiece}, for solo viola, \nameref{sec:bassPiece}, for solo contrabass, \nameref{sec:celloPiece}, for solo cello, and \nameref{sec:violinPiece}, for violin. 
These works all deal with different facets of the techniques that I have researched in this exegesis. 
Through the process of practice-based research and reflection on the implementation of these pieces, it is hoped that a clear identity of idiomatic treatment of these techniques will emerge.
Shortcomings are still valuable data points due to the lack of information readily available about the treatment and usage of these techniques.
It is envisaged that these works are to be used as etudes, both for musicians as testing grounds for the capabilities of the techniques, and for composers seeking to study scores to better understand how to implement these techniques in their own works.
Through the process of journalling my compositional intent, it will become clear what the function of each piece is.
Comparisons and contrasts to pre-existing literature and works will support the acceptance of the techniques as idiomatic.
I worked with several instrumentalists proficient in their instruments at a professional level, but were unfamiliar with the techniques.
They sight-read the works, and provided feedback on the writing, the technique, and how to better produce the techniques.
Participants intimately familiar with the techniques were not approached both due to their unavailability, and because players unfamiliar with the techniques better fit the purpose of this exegesis: 
to inform composers and performers unfamiliar with the techniques to the point where they are able to accurately replicate the techniques.

Because these were sight-reading sessions, and due to the scope of the research, the names of the participants have, by agreement, been altered to preserve their anonymity.\footnote{This is because the participants had limited rehearsal time, and 
the techniques are not representative of their normal playing.}
Their pseudonyms and instruments are as follows:

\begin{itemize}
  \item Angus Appleseed --- viola
  \item Jane Smith --- cello
  \item Joe Bloggs --- contrabass
\end{itemize}

Due to the scope of this exegesis and time constraints, it was not possible to obtain recordings of each piece in full.
Excerpts have been recorded where possible, and are supported by examples of other works from the literature, and instructional videos that support the idiomacy of the treatment of the techniques.
As the goal of the exegesis is to produce a practical document that future composers can refer to, the entirety of the scores are included in the appendices of this exegesis.
To aid the reader in referencing the relevant document quickly, hyperlinks are provided initially to the corresponding scores. 
Excerpts are provided for further references to the scores.

% \subsection{Background}
% Implement the techniques in a musical context.
% \subsection{Research statement/problem}
% Compositions will show both how these techniques can be used idiomatically, and how they can inform my craft.
% \subsection{Aim and scope of thesis}
% Writing works which will increase the collective understanding of how to implement these techniques.
% \subsection{Significance of work}
% Incorporating these techniques into my compositional process will show the pitfalls and ways that these techniques can be used.

\section{\violinPiece}\label{sec:violinPiece}
\hyperref[app:violinPiece Score]{\violinPiece} is a solo work for violin that explores \hyperref[sec:halfHarmonicsDiscussion]{half-harmonics}.
It is a non-programmatic work, and the title was inspired by a question that my supervisor posed to me while I sought ethics approval for the exegesis.
Half-harmonics are perhaps one of the simplest techniques to achieve, produced by applying finger pressure halfway between that required to create a harmonic, and a \emph{normale} sound.
The scale of finger pressure is detailed in \autoref{tab:finger-pressure}.
\begin{table}[]
    \centering
    \caption{Finger Pressure \& Resultant Sound}\label{tab:finger-pressure}
    % \resizebox{\textwidth}{!}{%
    \begin{tabular}{@{}ll@{}}
    \toprule
    Finger pressure & Result        \\ \midrule
    Open            & Fundamental   \\
    Touching        & Harmonic      \\
    More pressure   & Half-harmonic \\
    Fingerboard     & Normale       \\ \bottomrule
    \end{tabular}%
    % }
    \end{table}

The one-dimensional nature of this facet of the techniques leaves little variability in the implementation of the technique. 
Thus, \violinPiece\space explores the relationship between half-harmonics and other finger pressures. 
Rapid change between half-harmonics, regular harmonics, and \emph{normale} makes the work an exercise in finger control, as well as an introductory work to half-harmonics.
I demonstrate the rapid changes of the half-harmonics in \autoref{fig:Excerpt from what are you doing with the humans, mm. 5}.

% TODO: Add in picture of half-harmonics  - https://trello.com/c/cYTkFAaV/27-add-in-picture-of-half-harmonics
\begin{figure}
    \includegraphics[width=\linewidth]{./resources/violinHalfHarmonicsExcerpt5.pdf}
    \caption{mm.\@ 5--7 of \violinPiece}\label{fig:Excerpt from what are you doing with the humans, mm. 5}
  \end{figure}

In \autoref{fig:violin31}, I use the D string to provide additional available harmonics; the fifth D string harmonic, octave D string harmonic, and a fourth artificial harmonic using a stopped A are all readily available underneath the violinist's fingers.
\begin{figure}
  \includegraphics[width=\linewidth]{./resources/violinHalfHarmonicsExcerpt31.pdf}
  \caption{mm.\@ 29--36 of \violinPiece}\label{fig:violin31}
\end{figure}

Thus, the difficulty is not in the fingering on the fingerboard, but the quick changes between harmonic, half-harmonic, \emph{normale}, and artificial harmonic.
Through this facet of eliminating needless complexity, the work serves as an etude targeting specifically the production of half-harmonics.

\violinPiece\space takes inspiration from Sciarrino's \hyperref[fig:sciarrinoExcerpt]{fifth Caprice}, and serves as a stepping stone to the more difficult work.\autocite[]{sciarrinoCapricciViolino1976} 
In practice, the pitch content of my work was obscured by the noisy texture of half-harmonics, rendering large portions of it as noise. 
With the knowledge that properly performed, the harmonic content would be obscured, I leant into this, writing in a more tonal style than I do usually.

I opted to omit specific string instructions for the majority of the work, as this work is at a difficulty level where players should be able to make intelligent decisions over which strings to use.
Where a natural harmonic is desired specifically, and not an artificial harmonic, the fundamental has been notated with a small notehead in parentheses. 

As the technique is available on both nodal points, and non-nodal points (i.e.\ the technique is not limited to the fingering positions of harmonics on each string), I wanted to clearly delineate this with a section comprised of half-harmonics that do not excite any harmonics, shown in \autoref{fig:halfHarmonicNotOpen}. 
The timbre of these non-nodal half-harmonics is darker than the others, and they are more difficult to speak.\autocite[]{smithFeedbackCelloSightreading2019}
\begin{figure}
  \includegraphics[width=\linewidth]{./resources/halfHarmonicNotOpen.pdf}
  \caption{mm.\@ 73--79 of \violinPiece}\label{fig:halfHarmonicNotOpen}
\end{figure}


I explore the inverse of this, playing on nodes to bring out the harmonic content as seen in \autoref{fig:halfHarmonicOpenStringsExcerpt}.
Here, \emph{normale} open string notes (for the most part) signal the end of the use of the string; in this way, the player will learn how to make the half-harmonics speak properly while moving the bow to different strings.

\begin{figure}
  \includegraphics[width=\linewidth]{./resources/halfHarmonicOpenStringsExcerpt.pdf}
  \caption{mm.\@ 24 of \violinPiece}\label{fig:halfHarmonicOpenStringsExcerpt}
\end{figure}

Another facet of half-harmonics that I wanted to explore was the timbral aspect of them, bar 27 shown in Figure~\ref{fig:violinHalfHarmonicsExcerpt27}.
Similar to a multiphonic in their rich harmonic content, half-harmonics do not have the purity of tone that regular harmonics have, and are more often a type of multiphonic, as evidenced by Fallowfield's example.\autocite[]{fallowfieldCelloMapHalf2013}
% TODO: Fallowfield half-harmonics example - https://trello.com/c/A4Awcmxz/30-fallowfield-half-harmonics-example
I explore the interplay of half-harmonics, regular harmonics, and \emph{normale}, double stopped minims and semibreves slowly changing from one mode of pressure to another.
In this way, violinists that play \violinPiece\space will become familiar with different modes of pressure played concurrently.

\begin{figure}
    \includegraphics[width=\linewidth]{./resources/violinHalfHarmonicsExcerpt27.pdf}
    \caption{Excerpt from \violinPiece}\label{fig:violinHalfHarmonicsExcerpt27}
  \end{figure}

Notationally, the rapid changes between half-harmonics and other techniques were an ideal testing ground for different notation types.
Notably, the issue of half-filled diamond noteheads not having any distinction between crotchets and minims is not an issue due to the bulk of the work dealing in smaller divisions of the beat.\footnote{Which is to say, that the connection of stems make the timing obvious for quaver and smaller noteheads, regardless of the notehead.}
The considerations of how to best notate half-harmonics are discussed further in \autoref{sec:notation-half-harmonics}.

Ultimately, I opted to implement the half-filled diamond noteheads, finding them to most accurately present the information in a way that was unobtrusive, and built upon pre-existing notation.
It should be noted that this is contrary to the opinion set out in Dimpker's seminal thesis on notation, \emph{Extended notation. The depiction of the
unconventional}, but conforms to Gould's opinion on the matter.\autocites[120--121]{dimpkerExtendedNotationDepiction2012}[61]{gouldBars2011}
% TODO: fix gould citation - https://trello.com/c/gYLN8lqN/28-fix-gould-citation


\section{\violaPiece}\label{sec:violaPiece}
% TODO: Write Doppelganger
\hyperref[app:violaPiece Score]{\violaPiece} is a piece for solo viola, written to explore the lower register of the viola using \hyperref[sec:subharmonicsDiscussion]{subharmonics} juxtaposed with upper harmonics. 
The pitch distance between subharmonics and harmonics sidestep the viola's usual role occupying the middle register.



\subsection{Findings of \violaPiece}

Workshopping an early draft of \violaPiece\space with \violaParticipant, I found that subharmonics came fleetingly, and were prone to `jump back' to the fundamental.\autocite[]{appleseedFeedbackExploratorySession2019}
This resulted in a rewrite, redirecting the focus of the technique as a more textural element. 
As such, I amended the score to treat subharmonics largely as a `special effect', and not ascribe importance to their pitched content.

A second workshop with \violaParticipant\space was more productive, and we found that the subharmonic technique spoke much more readily with a slower bow speed, combined with a consistent amount of pressure.\autocite[]{appleseedFeedbackSightreadingSession2019}
We found that the A string of the viola did not respond nearly as readily to the technique, to the point of being unusable.
The D and G strings both spoke acceptably, but due to the pressure required, often resulted in unwanted double-stopping.
The C string spoke very freely, with the technique coming readily.
Contrary to Fallowfield stating `It is very difficult to sustain the tone, which often has a high noise component', we found that with practice, the subharmonic was able to stabilise to the point of it being usable for longer sustained notes.\autocites[http://www.cellomap.com/index/the-string/plucking-striking-and-bowing-the-string/how.html]{fallowfieldCelloMapHandbook2009}[]{fallowfieldCelloMapExample2013}[]{appleseedFeedbackSightreadingSession2019}

To `ease the player into' the technique, \violaPiece\space begins with an open C string, which drones for a few seconds before crescendoing while moving the bow towards the fingerboard and slowing the bowing speed down, shown in \autoref{fig:doppelgangerStart}.
\begin{figure}
  \centering
  \includegraphics{./resources/doppelgangerStart.pdf}
  \caption{mm.\ 1--3 of \violaPiece.}\label{fig:doppelgangerStart}
\end{figure}
The pressure used to play \emph{sul ponticello} is the same as the pressure needed to play subharmonics, priming the player to only need to concentrate on the bow speed and location on the string.\autocite[]{appleseedFeedbackSightreadingSession2019}
This technique of applying the thought process of playing \emph{sul tasto}, but actually playing near the fingerboard appears to be an ideal way to guide players unfamiliar with the technique into achieving it without lengthy in-person demonstrations.\autocites[]{appleseedFeedbackSightreadingSession2019}{bloggsFeedbackContrabassSession2019}

\section{\celloPiece}\label{sec:celloPiece}

\hyperref[app:celloPiece Score]{\celloPiece} is an exploration in multiphonics, but is not strictly an etude.
I discovered several facets of the technique that were not previously discussed in the literature found in Fallowfield's work.\autocite[]{fallowfieldCelloMap}
I wanted to explore multiphonics as I explored \hyperref[sec:half-harmonics]{half-harmonics} in the piece \autoref{sec:violinPiece}, shifting between \emph{normale} and the technique.
Unfortunately, while workshopping with \celloParticipant, I found the non-binary nature of multiphonics precluded this from being a possibility; the multiphonics speak on a sliding scale, and controlled attempts to shift between the two sound as though they are just failing to speak.\autocite[]{smithFeedbackCelloSightreading2019}


For each pair of nodes that produce multiphonics, the lower of the two is easier to pitch due to the logarithmic correlation between string length and pitch;
that is to say, because the gaps between semitones are correlationally larger as the pitch lowers, the pitching can be more precise.

Double-stopped multiphonics are feasible, but because the angle of attack can impact the partials the multiphonic produces, the shift between single strings and double stopping can stop the multiphonic from speaking clearly.\autocite[]{smithFeedbackCelloSightreading2019}
Aurally, double stopped multiphonics seem to be more effective with the \emph{normale} note on a lower string than the multiphonic.



\section{\bassPiece}\label{sec:bassPiece}
% TODO: Write The Veldt
Inspired by the eponymous short story by Ray Bradbury, \hyperref[app:bassPiece Score]{\bassPiece} is a composition for solo contrabass that explores both \hyperref[sec:multiphonicsDiscussion]{multiphonics} and \hyperref[sec:subharmonicsDiscussion]{subharmonics} in the context of a soundworld; the musical language is derived from the harmonic series, exploiting the resonance of the bass to maintain a drone.\autocite[]{bradburyVeldt1951}
Similarly like the plot of Bradbury's work, this world is filled with danger but also beauty. 
This is reflected in my work in the use of subharmonics and multiphonics; difficult and fragile techniques that can break at a moments notice, inspired by Thurley's treatment of multiphonics in \emph{yet another example of the porousness of certain borders}.\autocite[]{thurleyAnotherExamplePorousness2014}
My intent with Veldt was to create a harmonic language and space that the performer was able to `roam around' in, and features several sections of improvisation or stochastic aleatory based around the harmonic series, representing the possibilities of the veldt.

\subsection{Findings of \bassPiece}
Writing for contrabass and experimenting on my own bass, I found subharmonics came most easily on the G string. 
This was confirmed in a workshop with \bassParticipant, although he was able to produce subharmonics on all other strings, too.
Subharmonics on the lower strings did not speak as well, and it was difficult to discern the lower frequency's shifts.
Because of this, \bassPiece\space uses the technique as a textural tool, rather than a melodic device.
% They translate well into animalistic 

Working with \bassParticipant, he identified the need for subharmonics to be played closer to the fingerboard than normal.
He theorised that the tension being weaker as the contact point moved away from the bridge, closer to the middle, made it easier to produce subharmonics via the raucous motion.\autocite[]{bloggsFeedbackContrabassSession2019}
A second stave displaying the music notated at actual pitch was necessary, as shown in \autoref{fig:multiphonicsBassExample}.
\begin{figure}
  \centering
  \includegraphics{./resources/multiphonicsBassExample.pdf}
  \caption{excerpt from \bassPiece\space showing the \emph{suono reale} stave.}\label{fig:multiphonicsBassExample}
\end{figure}
\begin{figure}
  \centering
  \includegraphics{./resources/multiphonicsFingeringExample.pdf}
  \caption{Excerpt from \bassPiece.}\label{fig:multiphonicsFingeringExample}
\end{figure}
In contradiction of Gould's recommendations, I opted to notate the \emph{suono reale} in treble clef, rather than treble transposed down an octave.\autocite[423]{gouldBars2011}
This is because the multiphonics activate partials well above those usually expected, resulting in the advice not being relevant.

To maintain ease of playing, I use similar fingerings across the strings to aid in the pitching of multiphonics, as seen in \autoref{fig:multiphonicsFingeringExample}.





% I was inspired greatly by Oliver Thurley's \emph{yet another example of the porousness of certain borders}, whose treat\autocite[]{thurleyAnotherExamplePorousness2014}
