% app1.tex (file to switch to appendix mode)
\newpage

% \invisiblechapter{\violinPiece}
\chapter[\violinPiece]{}


\vspace*{3cm}
\begin{center}
\textsc{for solo violin}
\vspace*{3.5cm}

\HRule{0.5pt}


\LARGE \textbf{\uppercase{\violinPiece}}
\HRule{2pt}

\vspace{1.3cm}

\normalsize October, 2019
\date{}

\vspace*{5\baselineskip}

Rhys Gray

\end{center}
\newpage
\section*{Program Notes}
% \violinPiece\space is a solo work for violin that explores \hyperref[sec:half-harmonics]{half-harmonics}.
It is a non-programmatic work, and the title was inspired by a question that my supervisor posed to me while I sought ethics approval for my exegesis; a simple phrase laden with possible contexts, spurring the imagination to try and complete the meaning.

% It is, in a way, an etude, treating the half-harmonics in a way similar to those found in Sciarrino's \hyperref[fig:sciarrinoExcerpt]{\emph{6 Caprricio for violin}}. 
Half-harmonics are produced by applying left hand finger pressure halfway between that required to create a harmonic, and a \emph{normale} sound. 
The sound that is produced should be a mixture of the stopped string pitch, the harmonic pitch, and a resistant, slightly noisy quality.
% They are notated in the score as a half-filled diamond notehead.

\section*{Notation}
\begin{itemize}

    \item Half-harmonics are notated in the score as a half-filled diamond notehead.
    \item Arrows denote gradual transitions to the technique that the arrow is pointing to.\begin{itemize}
        \item Arrows between notes denote transitions between the types of notes (i.e.\ \emph{normale} to harmonic finger pressure.)
      \end{itemize}
    % \item sp denotes \emph{sul ponticello}.
    % \item msp denotes \emph{molto sul ponticello}.
    % \item similarly, st denotes \emph{sul tasto}, and mst denotes \emph{molto sul tasto}
\end{itemize}

\newpage\label{app:violinPiece Score}
\includepdf[pages=-,pagecommand={}]{./resources/compositions/violin.pdf}