% app2.tex (file to switch to appendix mode)

\chapter[\violaPiece]{}

\vspace*{3cm}
\begin{center}
\textsc{for solo viola}

\vspace*{3.5cm}

\HRule{0.5pt}


\LARGE \textbf{\uppercase{\violaPiece}}
\HRule{2pt}

\vspace{1.3cm}

\normalsize October, 2019
\date{}

\vspace*{5\baselineskip}

Rhys Gray

\end{center}
\newpage
\newpage

\section*{Program Notes}
\violaPiece\space is a piece for solo viola, written to explore the lower register of the viola using \hyperref[sec:subharmonics]{subharmonics} juxtaposed with upper harmonics. 

Subharmonics are achieved through precise control of torsional oscillation, which usually produces the sound of an amateur string player's heavy handed, slow bowing. 

To play subharmonics, one should place the bow at the 6th partial of the harmonic series of the fingered pitch, and bow with excessive pressure and an absolutely consistent speed. 
The increased pressure will distort the vibration of the string, producing a phase loop which, in turn, produces the subharmonic. 


The production of subharmonics can be aided by using older strings (which work better due to fats building up on the strings). 
Making a counter-clockwise half-twist in the string can also make it easier to produce octave and major second subharmonics (additional twists can help achieve lower subharmonics, at the expense of higher ones).

\section*{Notation}
\begin{itemize}

    \item Subharmonics are notated in the score using a square notehead for the fingering, with a small notehead at the desired resultant pitch. Where additional clarification is needed, `sh' is placed above.
    \item Arrows denote gradual transitions to the technique that the arrow is pointing to.\begin{itemize}
        \item Arrows between notes denote transitions between the types of notes (i.e.\ \emph{normale} to harmonic finger pressure.)
      \end{itemize}
    \item sp denotes \emph{sul ponticello}.
    \item msp denotes \emph{molto sul ponticello}.
    \item similarly, st denotes \emph{sul tasto}.
    \item n denotes \emph{normale}.
    \item Feathered beams denote a gradual acceleration to a tremolo.
\end{itemize}

\newpage\label{app:violaPiece Score}

% \includepdf[pages=-,pagecommand={},width=\textwidth]{resources/compositions/violin.pdf}
\includepdf[pages=-,pagecommand={}]{resources/compositions/viola.pdf}