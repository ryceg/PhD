% Turabian Formatting for Theses and Dissertations, 2018/08/06
%
% Developed using the turabian-formatting package (2018/08/01), available through CTAN: http://www.ctan.org/pkg/turabian-formatting
%
% Additional document class formatting options:
%
% raggedright: ragged right formatting without hyphenations
% authordate: support for the author-date citation style
% endnotes: support for endnotes

% document class handles the big-picture formatting
\documentclass[draft]{turabian-thesis}

% For Turabian citations, use the biblatex-chicago package
\usepackage[noibid]{biblatex-chicago}
\addbibresource{backmatter/works-cited.bib}


% inputenc handles unicode (i.e. non-ASCII characters, such as letters with umlauts). This may need changing, depending on what you're using to compile (LuaTex, pdftex, etc.)
\usepackage[utf8]{inputenc}

% graphicx handles graphics. Requisite for hyperref, don't worry about this.
\usepackage{graphicx}

% For lorem ipsum text. use \lipsum[5] to create placeholder text. Increase the number in the brackets for more text.
% Very handy to get an overview of how it's going to look, but also can make you feel unreasonably confident- beware!
\usepackage{lipsum}

% it's unwise to mess with anything in this section- all of these are great packages, and if you don't use them, then LaTex doesn't include them, so there's no reason to turn them off.
% I have included some search tags in the hope that they're useful. Refer to the package's documentation for further information about how to implement them.
%%%%%%%%%%%%%%%%%%%%%%%%%%%%%%%%%%%%%%%%%%%%%%%%%%%%%%%%%%%%%%%%%%%%%%%%%%%%%%%%%%%%%%%%%%%%%%%%%%%%%%%%%%%%%%%%%%%%%%%%%%%%%%%%%%%%%%

% fancyhdr does fancy headers.
\usepackage{fancyhdr}

% for comments
\usepackage{comment}

% microtype handles microscopic improvements.
\usepackage{microtype}

% cmap makes pdfs searchable.
\usepackage{cmap}
% % % % search pdf, text, highlight text


% csquotes handles quotations
\usepackage{csquotes}
% % % % quotes, quoting, quotation

% ellipsis handles ellipses.
\usepackage{ellipsis}
% % % % ellpisis, ellipses, ..., . . . 

% allows half and half in tables. 
\usepackage{diagbox}
% % % % diagonal tables, split table

% allows multiple rows in tables.
\usepackage{multirow}

% For fancier tables.
\usepackage{booktabs}

% xcolor handles adding coloured text.
\usepackage{xcolor}
% % % % coloured text, color, red text

% for captioning
% \usepackage{caption} % hypcap is true by default so [hypcap=true] is optional in \usepackage[hypcap=true]{caption}

% for referencing the titles of sections
\usepackage{nameref}
% % % % reference titles

% for appendices
\usepackage{appendix}

% for including pdf pages
\usepackage[final]{pdfpages}

% for including quotes
\usepackage{epigraph}

% for in-line code
\usepackage{listings}
% % % % code font
\usepackage{hyperref}
% for todos
\usepackage{fixme}
\fxsetup{
    status=draft,
    author=Rhys,
    layout=pdfnote, % also try footnote, inline, or pdfnote
    theme=signature
}
\definecolor{fxnote}{rgb}{0.94, 0.97, 1.0}


% hyperlinks! The option [hidelinks] makes it still black i.e. not obviously a hyperlink. Remove [hidelinks] to make it the classic hyperlink style.

% \hypersetup{
%     colorlinks = false,
%     linkbordercolor = {white},
%     urlcolor = {blue}
% }

%%%%%%%%%%%%%%%%%%%%%%%%%%%%%%%%%%%%%%%%%%%%%%%%%%%%%%%%%%%%%%%%%%%%%%%%%%%%%%%%%%%%%%%%%%%%%%%%%%%%%%%%%%%%%%%%%%%%%%%%%%%%%%%%%%%%%%

% Information for title page
\title{Writing Wrong Right}
\subtitle{An Investigation in Composing with Extended Techniques}
\author{Rhys Gray}
% \course{Doctor of Philosophy (F4D)}
\institution{University of Tasmania}
\department{CALE --- Creative Arts, Conservatorium of Music}
\location{Hobart, Tasmania}
\date{\today}

% define new commands i.e. variables i.e. pieces of music
\newcommand{\violinPiece}{\emph{what are you doing with the humans}}
\newcommand{\violaPiece}{\emph{doppelganger}}
\newcommand{\bassPiece}{\emph{the veldt}}
\newcommand{\celloPiece}{\emph{liminal}}
% aliases for performers to maintain anonymity
\newcommand{\violaParticipant}{Angus}
\newcommand{\celloParticipant}{Sarah}
\newcommand{\bassParticipant}{Joe}
\newcommand{\work}{Joe}

\newcommand{\phdTitle}{Writing Wrong Right: An Investigation in Composing with Extended Techniques}

% for how deep the ToC should go
% -1 part     1 section     3 subsubsection  5 subparagraph
%  0 chapter  2 subsection  4 paragraph
\setcounter{tocdepth}{2}

% this is where the document starts. Comment out things that you don't need.
\begin{document}
\frontmatter{}
\maketitle
\tableofcontents{}
% \listofillustrations{}

\mainmatter{}
\input{./components/rolesAndResponsibilities.tex}
% Resources necessary for the project to be conducted
% Funding/support being sought or secured
This is a relatively low-cost project, as we will be soliciting volunteers with a professional interest in the subject matter to act as our participants.
No additional funding is being sought at the time of submission.
\subsection{Brief Background}
In 2019, I completed my Honours exegesis at the University of Tasmania, titled `Harmonic Based Extended Techniques and their Compositional Applications', a study of three extended techniques applicable to stringed instruments; half-harmonics, subharmonics, and multiphonics.
I am interested in pursuing this line of research further, as I believe that there is still more to be learnt about the techniques that I have already covered, and that many other extended techniques lack the literature that composers can use to make informed decisions about their use. 
The purpose of this study will therefore be to explore extended techniques further, filling literature gaps and incorporating the techniques into my own artistic language.

Cellist and new music specialist John Addison has expressed an interest in developing his technique of `double touch' harmonics with me, which he describes as `[\ldots] where one engages two of the harmonic nodes on the same string in the same series simultaneously, which allows upper harmonics in the series that have never been stable nor dependable to become reliable and certain.'
% This discovery expands exponentially not only the facility of producing the said harmonics but when applied to double-stops (the production of tones on two strings simultaneously) it increases the available options from around 40 to 243.


His technical and theoretical knowledge will be useful in testing and conducting practical research on the techniques, with eminent composer Sofia Gubaidulina stating 
    `I am convinced that we are dealing with a brilliant artistic personality here. 
    It is to be expected that John Addison’s activities as an interpreter will have a vital influence on the next generation of musicians.'\footnote{Personal correspondance between John Addison and Sofia Gubaidulina.}
As a spectralist composer, I am primarily concerned with extended techniques that make use of microtonalities, exploit the harmonic series, and spatial acoustics.
The recent shift of the Conservatorium of Music to the Hedberg provides an exciting opportunity to conduct research using the variable acoustic panels, and the ways that they can be used in site-specific works. 
The scope of my research would therefore be surrounding the treatment of these extended techniques; a holistic review of the techniques from the view of a composer and performers will shed light on the way that the techniques can best be produced.

As a composer, exploring subharmonics, multiphonics, double touch harmonics, and other extended techniques is particularly exciting, as they are fertile ground for new and unique sounds that can be used to develop my musical identity.
% I plan for the resultant thesis to be a practical document that composers and artists can use as a reference manual for the production and implementation of the techniques in their own practices.
My research into underexplored techniques will broaden the performative and compositional palette available to artists. 
Through the documentation of my process in researching this technique, it will be catalogued and brought into the literature, facilitating further development.


Through a comprehensive review of how composers construct their frontmatter, guidelines to how new and experimental techniques can be communicated to performers will be developed.
This will lower the friction of learning new works, and promote the uptake of contemporary works.
This will be further aided by the development of a \LaTeX{} style which can quickly scaffold the relevant extended techniques for consistent and universal verbiage.

The resultant thesis, `Writing Wrong Right: Composing With Extended Techniques', will consequently be a practical document, suitable as a reference for artists interested in implementing extended techniques into their practice.


\section{Literature Review}
This PhD continues upon the previous research that I conducted during my Honours, and there is significant overlap with the two topics.
Therefore, the first item that is worthy of mention would be `Harmonic Based Extended Techniques and their Compositional Applications', which includes a ground-level review of the seminal literature in the field.\autocite{grayHarmonicBasedExtended2019}
However, due to the limited scope of the exegesis, there were significant omissions, so for the sake of completeness its contents will be reviewed under the lens of `composing with extended techniques'.
Additionally, there are a number of sources that were either missed or cut from the initial literature review for sake of brevity.
% There appears to be a great deal of activity in similar spheres of research in Basel, Switzerland, and there are several manuscripts and texts that appear only in German.

Unlike standard pedagogical models, wherein the student learns from the teacher, the methods used in this exegesis are a more collaborative, and less rigid practice.
Composers typically will have a performer in mind when they compose a work, and collaborate with them, working to find what works for the artists and instrument.
However, in subsequent performances, that direct connection to the composer is often unavailable, and the performer is left to interpret the paratext, without direct instruction from the composer.
Paratextual instruction can vary in degrees of specificity, and will often be printed in the frontmatter of the work.
Where there is a well established understanding of the technique (such as the Bartok pizzicato technique, which has been accepted into the canon of `standard' techniques, and is no longer considered extended), this is not an issue, but can pose issues in the case of techniques where there is a great deal of variance possible, such as multiphonics and subharmonics.
In some cases, the performer must rely on previous works to build a contextual language in which to interpret esoteric markings, or use previous recordings of interviews and performances to ascertain the composer's intent.
Understanding the composer's intent is a crucial part in the preparation process, and the ability to accurately reproduce the composer's intent may negatively impact a work's longevity in the literature.
Many academics have recognised the deficit in training in extended techniques, both in their base form and their composer-specific implementation (which may vary from composer to composer, or piece to piece). 
Violist Sarah Wei-Yan Kwok discusses this in her thesis `Breaking the sound barriers: extended techniques and new timbres for the developing violist', where she commissions six etudes for viola exploring extended techniques, along with an investigation into pedagogy.\autocite{kwokBreakingSoundBarriers2018}

The composer-performer collaboration dynamic has been explored at length; the most famous example being the virtuoso pianist Paul Wittgenstein, whose career came to a halt during World War 1, when he lost his right arm.
Afterwards, he began commissioning leading composers to write for him, resulting in Richard Strauss', `Parergon zur `, `Sinfonia Domestica for Piano and Orchestra', Maurice Ravel's `Piano Concerto for the Left Hand in D major', and further works written by Hindemith, Korngold, Britten, and Prokofiev, to name a few.\autocite[107]{predotaPaulWittgensteinVoice2014}
Collaboration with the intended performer is crucial to a composer's ideas being realised accurately, as articulated by Jack Barnes, who stated:
\begin{quotation}
    My collaboration with Mathieson-Sandars allowed for subtle but important improvements to \emph{Lines, Contexts and Freedoms}. 
    Certain aspects such as the composer hearing his piece for the first time, experimenting with different pianistic timbres and creating more comfortable hand distributions were among the most useful outcomes of our collaboration. 
    As the performer, it was useful for my interpretation to learn [the composer's] intent behind the gestures; that the effect of dynamic shaping was not desirable for all of them.\autocite[20]{jackbarnesExaminationComposerPerformerCollaborations2017}
\end{quotation}
Discussion being needed with the composer to learn the composer's intent suggests that there was information being provided to Barnes that was not present in the paratext.
Future performers of the work that was commissioned by Barnes may not have the same level of access to the composer, hence the need for a thorough and precise documentation of the intended sound being available.

Sheet music is an abstracted conception of a musical work; the printed paper is merely a set of instructions that are interpreted by the performer, in order to express a musical idea that only the composer truly knows.
Without a perfect abstraction of what the composer imagines, a `perfect' reproduction is impossible.
Indeed, even if there \emph{was} a perfect reproduction, the variance in instrumental timbre and other facets of the process render it a fool's errand.
The Platonic ideal is unattainable, and perhaps why little effort has been made to make further progress towards the liminal.\autocite{citation needed, obviously}

Attempts at combining notation with composer intent have been made, with Eftiha Victoria Arkoudis' `Contemporary Music Notation for the Flute: A Unified Guide to Notational Symbols for Composers and Performers' building on the work of Robert Dick.\autocite{arkoudisContemporaryMusicNotation2019}

Instructive manuals are typically in one of two categories; performer-oriented, in which the techniques presented are articulated in a performer-focused context, and composer-oriented; orchestration manuals which dictate the end results.
For brevity, these usually do not intersect, or at least are not written comprehensively with both the composer and performer in mind. 
The lack of literature that encompasses the entirety of the production of sound process, from the action to the resultant, is slowly being rectified, particularly with online resources such as Heather Roche's work in clarinet multiphonics, and Fallowfield's with CelloMap gaining popularity.\autocite{rocheHeatherRoche, fallowfieldCelloMap}
\subsubsection{Notation Manuals}
Notation has long been the domain of Kurt Stone and Gardner Read being the authoritative voices, writing extensively on the subject beginning in the 1970s.\autocite{stoneMusicNotationTwentieth1980,readCompendiumModernInstrumental1993,readContemporaryInstrumentalTechniques1976,readMusicNotationManual1979a}
Additional texts by David Cope and Erhard Karkoschka supported these with their own contributions.\autocite{copeNewMusicNotation1976,karkoschkaNotationNewMusic1972}
Editor of Faber Music, Elaine Gould's seminal `Behind Bars' gave an authorative voice to the methods of notating along with the building blocks for techniques which she felt had not been developed to the point of consensus.\autocite[iii]{gouldBars2011}

\subsection{Physics}
In order to be able to accurately prescribe instructions on how to produce these techniques, there must be an understanding of the underlying physics.
Helmholtz provides our basic undestanding of the stick---slip motion which makes a bow produce sound on a string.\autocite{helmholtzSensationsTonePhysiological1954}
Knut Guettler and Håkon Thelin have provided extensive research into how the string can produce multiphonics.\autocite{thelinAnalysisBowedStringMultiphonics,thelinMultiphonicsDoubleBass2011,guettlerBowedstringMultiphonicsAnalyzed2012,guettlerGuideAdvancedModern1992,guettlerWaveAnalysisString1994}
\fxnote{This needs to be expanded}{R. T. Schumacher, Kenneth Marskall, and Anders Askenfelt contribute additional aspects to the literature.}\autocite{schumacherTransientBehaviourModels1995,askenfeltMeasurementBowingParameters1989,marshallModalAnalysisViolin1985}
For subharmonics, the violinist that discovered the technique, Mari Kimura, has contributed practical examples on how to produce them.\autocite{kimuraHowProduceSubharmonics1999}
\fxnote{Empirical data on what? Expand!}{However, for research into the technique's production, Robert Mores, Guettler, Erwin Schoonderwalt, Carleen Hutchins, and John Cantrell provide empirical data.}\autocite{moresFurtherEmpiricalData2019,guettlerBowedStringDevelopment2002,schoonderwaldtViolinistSoundPalette2009,hutchinsSubharmonicsPlateTap1960,cantrellSubharmonicGenerationChaos2015}

\subsection{Extended Techniques}
To understand extended techniques, we must first establish a basis of what is considered a `regular' technique, or rather, what the qualifying factors are for a technique to be considered extended.
To do this, we will look at what techniques are commonplace in the literature, and which are less so.
Techniques that require descriptions in the frontmatter or otherwise `extend' the instrument beyond the normal canon would reasonably be understood to be considered extended techniques.
Read discusses this, stating:
\begin{quotation}
    Many so-called ‘new’ instrumental devices have developed from well established techniques; they are extensions of, or refinements of, procedures long considered part of a composer’s repertorium of expressive devices. 
    The newness, then, is not one of kind but of degree, a further and more extensive development of basic effects found in scores from the late nineteenth century to the present day.\autocite[3]{readContemporaryInstrumentalTechniques1976}
\end{quotation}
Unpacking the language that we use, extended techniques are just that; an extension of the traditional techniques that are already established in the canon. 

\fxnote{This probably needs retooling to fit better in the literature review.}There are three aspects that are relevant to the literature review; literature surrounding the techniques themselves, literature with how the techniques are presented, and scores. 
Scores are necessary to understand how composers implement extended techniques in practice.
By reviewing their frontmatter and the symbol notation, we gain an understanding of how composers present information to performers, and can build a system of notation that maps consistently with the rest of the established canon.
Consequently, seminal scores will be included in the literature review so that we may understand how they work. 
These scores' notation systems for extended techniques will be broken down into their elements, and through comparative analysis of how different composers implement the same techniques, we will be able to find how textual, symbolic, and graphic notation systems can be used to achieve the desired results.

Dimpker postulates the following criteria for extended technique notation; \begin{quote}
    `(1.) As exact as possible and (2.) As simple as possible while the system may (3.) Not be contradictory to
traditional notation, but should instead extend and be closely related to it.'\autocite[3]{dimpkerExtendedNotationDepiction2012}
\end{quote}
This meshes well with the philosophy of Elaine Gould's `Behind Bars', which advocates for a consistent style language, despite the two texts occasionally coming at odds with one another in the exact treatment of some cases.\autocite[120--121, 61]{dimpkerExtendedNotationDepiction2012,gouldBars2011}
Dimpker posits that four methods of notation are acceptable in order to realise exact and precise notation; `action notation, symbolic notation, diagrammatic notation, and schematic notation'.\autocite[33]{dimpkerExtendedNotationDepiction2012}

Textual notation, i.e.\ instructions printed in the score, are the most straight forward, but limited to what can be summarised in few words.
Symbolic notation assigns the technique to a symbol, typically with the instructions placed in the frontmatter.
Graphical notation systems are often used when the binary of symbolic notation is restrictive, and requires a greater fidelity than textual notation can provide. 
An example of this can be seen in Kaija Saariaho's notation of overpressure, wherein she uses a black bar to represent the amount of overpressure required, temporally relational to the position in the score.\autocite{TODO:saariaho citation}

\subsubsection{Double Touch Harmonics}
The technique of `double touch harmonics' as described by John Addison appears to be almost totally novel; it may be a matter of the technique being developed under a different name, but the only reference that seems to be relevant is a 1980 thesis, which has not been made available online.\autocite{woodrichMultinodalPerformanceTechnique1980}
Addison has completed several fingering charts and made preliminary instructive documentation on the technique, which has been provided to the researched.
Due to the lack of literature, collaboration with performers will be essential in order to build the technique to a stable, usable state.

\subsubsection{Multiphonics}
Multiphonics have been well-documented in reed instruments, with Robert Dick's seminal `The Other Flute', setting the gold standard in extended technique documentation, painstakingly notating the outputs, fingerings, and qualities of flute multiphonics and other extended techniques.\autocite{dickOtherFlute1989}
There are several Barenreiter technique manuals for other aerophones that cover multiphonics, including bassoon, saxophone, violin, and oboe, with the publisher appearing to try and compile a manual for every instrument.\autocite{weissTechniquesSaxophonePlaying2010,galloisTechniquesBassoonPlaying2009,ardittyTechniquesViolinPlaying2013, TODO:OboeBook}
\fxnote{Double check to see if this is true}The book on violin by violinist Irvine Arditty notably lacks any information on multiphonics, perhaps due to the difficulty of production on the small instrument.\autocite{ardittyTechniquesViolinPlaying2013}

\paragraph{String instruments}
The development of literature dealing with multiphonics on stringed instruments is more recent, with a special January 2020 Tempo journal issue, dedicated entirely to string multiphonics.
It was collated by Dr.\ Ellen Fallowfield, who notably contributed the landmark thesis CelloMap (and eponymous website) in 2009.\autocite{fallowfieldCelloMapHandbook2009, fallowfieldCelloMap}
Her article, `Cello Multiphonics: Technical And Musical Parameters' expands upon the work that she began in her thesis, `CelloMap'.\autocite{fallowfieldCelloMultiphonicsTechnical2020}


\paragraph{Plucked chordophones}
Rita Torres has explored multiphonics on the guitar extensively, with several papers dedicated to their research, including her thesis, `A New Chemistry Of Sound: The Technique Of Multiphonics As A Compositional Element For Guitar And Amplified Guitar'.\autocite{torresMultiphonicsCompositionalElement2012}
Paulo Ferreira-Lopez and Torres provide a robust exploration of previous systems of notation of multiphonics, suggesting notational methods similar to those that Fallowfield suggests.\autocite{ferreira-lopesGuitarMultiphonicsNotations}
Catalogues of multiphonics in the guitar literature show that there has been little interest in the technique, with the first score dating back to 1832, but only around twenty or so pieces composed since the techniques inception.\autocite[80--82]{torresSoundWorldGuitar2018}
It has been suggested that the widespread adoption of the technique has been marred by the mode of playing naturally having a decay, making multiphonics more difficult to hear for an audience, as well as the guitar having a poor dynamic range.\autocite[21--22]{Torres2014TowardsOT}
This has been all but confirmed with the analysis of multiphonic signal decay at distances equivalent to that of an audience revealing that without amplification, the multiphonics will be inaudible.\autocite[279]{torresGuitarMultiphonicsInfluence2014}
Thomas Ciszak and Seth Josel expand upon the research carried out by Torres, collating a catalogue of performable multiphonics from strings 3 --- 1, and examining multiphonics on the electric guitar.\autocite{ciszakNeonLightMultiphonic2020}

\paragraph{Piano}
The piano, an unlikely candidate for multiphonics, has had some success in making inroads in the technique through the use of preparing the piano.
With the development of more robust literature, the prospect of developing a new sound-palette for the piano is alluring.
Interest begins in 2016 with Juhani Vessikala's MA thesis on the topic.\autocite{vesikkalaMultiphonicsGrandPiano2016}
Sanae Yoshida acknowledges Vessikala's work in an article in the multiphonics issue of Tempo, where Yoshida expands upon the practical ways in which a composer can use microtonality (and multiphonics) on a piano.\autocite{yoshidaMicrotonalPianoTunedIn2020}
Following is Caspar Walter's algorithm to calculate the frequency components of pure multiphonics.\autocite{casparjohanneswalterVariantsContinuedFraction2019}


\input{./components/rationale.tex}
\subsection{Key Questions}
\begin{itemize}
    \item How are extended techniques used in current literature, and are there ways to improve their delivery and make them more accessible to composers, performers, and audiences?
    \item Are there extenuating circumstances that keep these techniques from entering mainstream literature, or are they simply still in their infancy? 
    \item How have other artists used these techniques, and what can we learn from artists that have already incorporated them into their practice?
    \item How can I incorporate these extended techniques into my personal practice and develop a unique style with them?
    \item How well understood is the physical production, and are there ways we can improve production of the sound in a performance context?
    \item What variables impact the production of these techniques?
    \item What can we learn from the way that these techniques are physically produced? 
\end{itemize}


\subsection{Aims}
\begin{itemize}
   \item Develop my artistic voice and personal style through the incorporation of these extended techniques into my practice.
   \item Broaden the field of research by studying extended techniques that have not been extensively researched.
   \item Develop ways of communicating the best practices of techniques to increase their accessibility to others by formalising notation.
   % \item The best practice of how to produce the techniques will be synthesised by understanding the physical properties of the techniques and how they are produced. 
\end{itemize}


% My PhD is about building better music notation. 
% It can be thought of as taking an old house, and reconstructing blueprints of it, so that anyone can build extensions on it, and they know which walls they can knock down.
% I'm then building a catalogue of room designs that can replace parts of the existing structure, or be built as extensions;
% for composers that don't want to use rhythm as presented in Western staff notation, they can tear out those supporting beams, and replace it with 
% I'm looking at the Western staff notation that already exists, and breaking it down to its base elements, so I can construct building blocks for new notation out of that. 
% That, combined with my own set of rules that are developed from best-practice design and extrapolations from existing notation will help inform how to construct new techniques and notation. 
% The benefit of this is that it will help future-proof our notation for future technologies, 
\subsection{Outcomes Of This Project, and Why It Is Relevant}
This project will provide me with a better understanding of the mechanics and musical capabilities of the extended techniques. As these techniques currently have a deficit of literature, both instructive and artistic; there are few resources for people to learn from, and even fewer practical examples of how to implement the techniques in a musical context.
My research will address this, filling the research gaps where identified.
The outcome of my research into the technique of `double touch harmonics', as developed by John Addison, is particularly relevant as it presents an exciting possibility for a new method of producing familiar harmonics.
This will increase the number of available fingering positions of harmonics within existing compositions, as well as providing more colour options for performers to choose from.


\subsection{Methodology}
% Autoethnography, document analysis, interviews, research
Through interviews with players at varying stages of proficiency and familiarity with the techniques, I will be able to uncover the barriers to producing these techniques. 
Document analysis of existing resources and compositions will help direct and support the line of enquiry. 
Autoethnography of my creative process will document the research process and clarify my intent.  

To understand extended techniques, we must first establish a basis of what is considered a `regular' technique, or rather, what the qualifying factors are for a technique to be considered extended.
To do this, we will look at what techniques are commonplace in the literature, and which are less so.
Techniques that require descriptions in the frontmatter would reasonably be understood to be considered extended techniques.

The aim of this research project is not to make the techniques popular enough to make clarification of technique unnecessary, or for it to enter the canon of techniques so that it is no longer considered to be `extended' (as the Bartok pizzicato has).
Rather, this is intended to act as a resource for composers and artists to be drawn upon as a reference for when they wish to use the technique.
A considered and informed judgement call over a technique can only be made when the technique is understood well.
The composer will communicate the information necessary to realise the technique to the player, typically through the frontmatter. 
In order to better understand what information composers deem useful to communicate to players, a review of scores with similar techniques will take place.
By breaking the score's frontmatter content up into its actions, we can understand how composers communicate their desired techniques to players.

% Description and number
% Inclusion and exclusion criteria
% Sample size and statistical or power issues

Participants will be recorded with both video and audio for both workshopping pieces and interviews regarding the pieces, which may be included as part of the exegesis. 
If a participant wishes to not have their name attached to the research, video would be omitted from the pieces, and the interviews would be transcribed.
% Participant recruitment strategies and timeframes (as required in addition to that outlined in the HREA)
\paragraph{Recruitment Email}

Below is an example of the email that will be sent to potential participants.

\begin{quotation}
    To whom it may concern,

My name is Rhys Gray, and I am a PhD student at the University of Tasmania’s Conservatorium of Music. 
My research is focused around under--researched extended techniques, and their compositional applications. 
I am currently seeking interested participants that play an instrument at a professional level that would be willing to assist me. 
This would involve workshopping pieces for your instrument, and being interviewed about the pieces, techniques used, and process. 
These pieces would include extended techniques, such as multiphonics, overtones, subharmonics, and other non-standard methods of producing a sound on your instrument. 
This would be suitable for a musician that is interested in exploring new techniques, and collaborating with a composer on new works. 
Please note that this research has the potential to be personally identifiable, or, with consent, directly identify you if necessary. 
If you are interested, please contact me at rhys.gray@utas.edu.au with your name, instrument, and availability for further discussion.

\end{quotation}
% Approach/es to provision of information to participants and/or consent (as required in addition to that outlined in the HREA)
% If necessary, the type, timing and context of consentsought from different participant groups, and any arrangements to confirm or re-negotiate that consent.
% If necessary, details of who will be confirming or re-negotiating consent with participants and the process/es that will be undertaken.
% Research Activities: What you are going to do?
% Participant commitment
% Project duration
% Participant follow-up
% Data Collection/Gathering: What information are you going to collect/gather? (as required in addition to that outlined in the HREA)
% Data collection/gathering techniques: How will you collect/gather the information?
% Impact of and response to participant withdrawal
\subsubsection{Data Collection}
We will be gathering audiovisual data of participants with cameras, microphones.
Participants will be able to withdraw from the study at any time without consequence. 
Withdrawal of a participant
% Data Management: How will you store, provide access to, disclose, use/re-use, transfer, destroy or archive the information that you collect/gather? (as required in addition to that outlined in the HREA)
% Include a data management plan in accordance with National Statement 3.1.45 and 3.1.56
% Data Analysis: How will you measure, manipulate and/or analyse the information that you collect/gather?
% Matching and sampling strategies
% Accounting for potential bias, confounding factors and missing information
% Statistical power calculation
% Data Linkage: What linkages are planned or anticipated?
\input{./components/outcomeMeasures.tex}
% Plans for return of results or findings of research to participants
% Include an ethically defensible plan in accordance with National Statement 3.1.65 or 3.2.15 or 3.3.36–3.3.61, as appropriate
% Plans for dissemination and publication of project outcomes
% Other potential uses of the data at the end of the project
% Project closure processes
% Plans for sharing and/or future use of data and/or follow-up research
% Anticipated secondary use of data

% \subsection{Outputs}
My research into underexplored techniques will broaden the performative and compositional palette available to artists. 
Through the documentation of my process in researching this technique, it will be catalogued and brought into the literature, facilitating further development.


Through a comprehensive review of how composers construct their frontmatter, guidelines to how new and experimental techniques can be communicated to performers will be developed.
This will lower the friction of learning new works, and promote the uptake of contemporary works.
This will be further aided by the development of a \LaTeX style which can quickly scaffold the relevant extended techniques for consistent and universal verbiage.

The resultant thesis, `Writing Wrong Right: Composing With Extended Techniques', will consequently be a practical document, suitable as a reference for artists interested in implementing extended techniques into their practice.

\backmatter{}

\begin{appendixes}
    % % app0.tex (file to switch to appendix mode)
% No need to alter this file...
\newpage
\newcommand{\HRule}[1]{\rule{\linewidth}{#1}}
\newcommand\invisiblechapter[1]{%
  \refstepcounter{chapter}%
  \addcontentsline{toc}{chapter}{\protect\numberline{\thechapter}#1}%
  \chaptermark{#1}}
\appendix\appendixpage\addappheadtotoc\space


\end{appendixes}

\printbibliography{}

\end{document}

% Project Team Roles & Responsibilities

% Names, affiliations, positions and responsibilities of investigators and other key project team members (as required in addition to that outlined in the HREA)
% Identification of research sites or collaborating/contributing services or individuals not otherwise included in the HREA.
% Resources

% Resources necessary for the project to be conducted
% Funding/support being sought or secured
% Background

% Literature review
% Rationale/Justification (i.e. how the research will fill any gaps, contribute to the field of research or contribute to existing or improved practice)
% Research questions/aims/objectives/hypothesis
% Expected outcomes
% Project Design

% Research project setting (physical sites, online forums and alternatives)
% Methodological approach
% Rationale for choices of method/s (tied to project aims/objectives)
% Rationale for the choice of any control arm
% Participants
% Description and number
% Inclusion and exclusion criteria
% Sample size and statistical or power issues
% Participant recruitment strategies and timeframes (as required in addition to that outlined in the HREA)
% Approach/es to provision of information to participants and/or consent (as required in addition to that outlined in the HREA)
% If necessary, the type, timing and context of consentsought from different participant groups, and any arrangements to confirm or re-negotiate that consent.
% If necessary, details of who will be confirming or re-negotiating consent with participants and the process/es that will be undertaken.
% Research Activities: What you are going to do?
% Participant commitment
% Project duration
% Participant follow-up
% Data Collection/Gathering: What information are you going to collect/gather? (as required in addition to that outlined in the HREA)
% Data collection/gathering techniques: How will you collect/gather the information?
% Impact of and response to participant withdrawal
% Data Management: How will you store, provide access to, disclose, use/re-use, transfer, destroy or archive the information that you collect/gather? (as required in addition to that outlined in the HREA)
% Include a data management plan in accordance with National Statement 3.1.45 and 3.1.56
% Data Analysis: How will you measure, manipulate and/or analyse the information that you collect/gather?
% Matching and sampling strategies
% Accounting for potential bias, confounding factors and missing information
% Statistical power calculation
% Data Linkage: What linkages are planned or anticipated?
% Outcome measures